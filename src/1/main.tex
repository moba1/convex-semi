% upLaTeX用にuplatexオプションが指定してあるが、
% pLaTexの場合ははずす
\documentclass[a4paper, 10pt, uplatex]{jsreport}

\usepackage[T1]{fontenc}
\usepackage[utf8]{inputenc}
\usepackage[backend=biber, maxnames=100, backref=true]{biblatex}
\usepackage[binary-units = true]{siunitx}
\usepackage{amsmath, amssymb, amsthm}
\usepackage{ascmac}
\usepackage{graphicx}
\usepackage{hyperref}

\newtheorem{example}{例}
\newtheorem{theorem}{定理}
\newtheorem{lemma}{補題}

\begin{document}
\chapter{イントロダクション}
\section{数理最適化}
数理最適化問題(または単に最適化問題)は以下の形で表される。
\begin{alignat}{2}
    &\text{minimize}   & \quad f_0(x) & \notag \\
    &\text{subject to} & \quad f_i(x) & \leq b_i, \quad i = 1,\cdots,m \label{eq:convex-define}
\end{alignat}
ここで、ベクトル$x = (x_1, \cdots, x_n)$は問題の最適化変数、$f_0 : \mathbb{R}^n \rightarrow \mathbb{R}$は問題の目的関数、$f_i : \mathbb{R}^n \rightarrow \mathbb{R}, \quad i = 1,\cdots,m$は(不等形の)制約関数、定数$b_1,\cdots,b_m$は制約関数の限界(または境界)である。
この問題の解、または最適値はベクトル$x^*$で表され、制約を満たす全てのベクトルの中で最小の目的値となる。
式で示すと、$\forall z$ with $f_1(z) \leq b_1, \cdots, f_m(z) \leq b_m$の時、$f_0(z) \geq f_0(x^*)$を満たすベクトル$x^*$が問題の解となる。
一般的には、目的関数と制約関数が特定の形式で特徴付けることが出来る最適化問題のクラスや族について考える。
重要な例として\ref{sec:least-square}節と\ref{sec:linear-problem}節で示す最小二乗法や線形計画問題がある。
例えば線形計画問題では、もし目的関数と制約関数$f_0,\cdots,f_m$が線形である場合、以下を満す。
\begin{align}
    f_i(\alpha x+\beta y) = \alpha f_i(x) + \beta f_i(y) \label{eq:linear}
\end{align}
ここで、$\forall x, y \in \mathbb{R}^n, \forall \alpha, \beta \in \mathbb{R}$である。
問題が線形ではない場合は非線形計画問題と呼ばれる(\ref{sec:nonlinear-problem}で少し触れる)。
本書で扱うのは凸最適化問題と呼ばれる問題であり、目的関数と制約関数が凸面である問題である。
凸面であるということは以下を満たすことを言う。
\begin{align}
    f_i(\alpha x+\beta y) \leq \alpha f_i(x) + \beta f_i(y) \label{eq:convex}
\end{align}
ここで、$\forall x, y \in \mathbb{R}^n, \forall \alpha, \beta \in \mathbb{R}$ with $\alpha + \beta = 1, \alpha \geq 0, \beta \geq 0$である。
\eqref{eq:linear}と\eqref{eq:convex}を比較すると、凸面は線形性の一般化であることが分かる。
\end{document}

